\documentclass[11pt,fullpage]{article}
%\usepackage{multicol,wrapfig,amsmath,subfigure}
\usepackage{multicol,amsmath,amssymb,algorithmic,bm}
\usepackage{graphics,graphicx}
\usepackage[right = 2.5cm, left=2.5cm, top = 2.5cm, bottom =2.5cm]{geometry}
\pagestyle{plain}
\def\urlfont{\DeclareFontFamily{OT1}{cmtt}{\hyphenchar\font='057}
              \normalfont\ttfamily \hyphenpenalty=10000}

%\input macros

\newcommand{\subheading}[1]{\noindent \textbf{#1}}
\newcommand{\grad}{\nabla}
\newcommand{\jump}[1]{[#1]}
\newcommand{\limit}[2]{\lim_{#1 \rightarrow #2}}
\newcommand{\mb}[1]{\mathbf{#1}}
\newcommand{\reals}[1]{\mathbb{R}^{#1}}


\begin{document}
\noindent
\begin{center}
{\bf .smesh File Format}
{\\ Fei Zhu \\ 06/10/2014}
\end{center}

\section{Introduction}

We introduce the file format for simulation mesh in Physika,
.smesh. ``smesh'' stands for ``simulation mesh'', it's not the
``.smesh'' format used by TetGen \cite{si2006}. The .smesh format is
heavily built on the .veg format developed by Jernej Barbi\v{c} in VegaFEM
\cite{vega}. Unlike VegaFEM, we do not include material properties in simulation
mesh files and .smesh files also support simulation mesh in
2D. Another feature of .smesh files is that non-uniform element types
are supported, i.e., the mesh can be composed of elements with
different number of vertices.

\section{.smesh File Format}

\textbf{Basics:} The smesh file format is ASCII. Lines starting with *
denote a command. Lines starting with
\# are comments. Empty lines are ignored. Files can be nested using
the *INCLUDE command. The effect of *INCLUDE is to include, at that
point in the .smesh file, the entire contents of the included
file. Include files can include other files; they can nest
arbitrarily.
\\\\
\noindent{}\textbf{Vertices:} Vertices are specified using the *VERTICES
keyword. The second line gives the number of vertices, followed by
integer ``2'' (two-dimensional simulation) or ``3'' (three-dimensional
simulation), optionally followed by more parameters, which are
ignored. The subsequent lines give one vertex per line, in the format:
\\\\
\noindent{}$<$vertex index$>$ $<$x$>$ $<$y$>$ [z]
\\\\
\noindent{}where the vertex index starts either at 0 or 1, increments by 1 for
every vertex. The unit for vertex positions is meters.
\\\\
\noindent{}2D example:
\\\\
\noindent{}*VERTICES\\
5 2 0 0\\
1 0.5 0.5\\
2 1.0 -0.5\\
3 -1.0 0.0\\
4 0.25 -0.25\\
5 0.6 0.2\\

\noindent{}3D example:
\\\\
\noindent{}*VERTICES\\
5 3 0 0\\
1 0.5 0.5 0.5\\
2 1.0 -0.5 0.5\\
3 -1.0 0.0 1.0\\
4 0.25 -0.25 0.5\\
5 0.6 0.2 0.3
\\\\
\noindent{}\textbf{Elements:} Elements are specified using the
*ELEMENTS keyword. The second line gives the element type, namely
``TRI'', ``QUAD'', ``TET'', ``CUBIC'', or ``NONUNIFORM''. ``TET'' and ``CUBIC'' stand
for tetrahedral and cubic mesh for 3D simulation, ``TRI'' and ``QUAD''
are their counterparts in 2D, ``NONUNIFORM'' means non-uniform elements. The third line gives the
number of mesh elements, followed by the number of vertices in each
element (3 for triangles, 4 for quads and tets, 8 for cubes, -1 for arbitrary), followed
by some optional integers that are ignored. Subsequent lines give one
element per line, in the format:
\\\\
\noindent{}$<$element index$>$ $<$vertex 1$>$ $\cdots$ $<$vertex n$>$
\\\\
\noindent{}where the element index starts either at 0 or 1 and
increments by 1 for every element. 
\\\\
\noindent{}Tetrahedron example:
\\\\
\noindent{}*ELEMENTS\\
TET\\
2 4 0\\
1 2 3 4 1\\
2 5 3 4 1
\\\\
\noindent{} specifies a mesh with two tets. The first tet has vertices
2, 3, 4, 1 (in that order), and the second tet has vertices 5, 3, 4,
1. The ordering of vertices within the element matters. Incorrect
order give wrong results.
\\\\
\noindent{}2D non-uniform elements example (assume the dimension has
been specified by vertices):
\\\\
\noindent{}*ELEMENTS\\
NONUNIFORM\\
3 -1 0\\
1 1 2 3\\
2 1 3 4 5\\
3 1 5 6
\\\\
\noindent{} specifies a mesh with three elements. The first element is
a triangle with vertices 1, 2, 3 (in that order), the second
element is a quad, and the third element is again a triangle.
\\\\
\noindent{}\textbf{Regions:} Regions are used to store indices of elements that share the same property,
for example material property. Regions of .smesh files are equivalent to
Sets of .veg files. They are specified using the *REGION
keyword, followed by the name of the region. The next lines are
comma-separated list of set elements. Elements should be stored and
identified by the indices given in the *ELEMENTS section, i.e., either
0-indexed or 1-indexed, depending on whether *ELEMENTS are given
0-indexed or 1-indexed. Example:
\\\\
\noindent{}*REGION region1\\
\noindent{}11,17,21,37,113,310,555,\\
\noindent{}556,557,570,601,991,1013,1210,1225
\\\\
\noindent{}\textbf{Easy re-use of Stellar/TetGen meshes:} The .smesh file format supports
easy re-use of Stellar/TetGen meshes since it's derived from .veg
file. Suppose that the mesh vertices and elements are stored in
myMesh.node and myMesh.ele files, in the standard Stellar/TetGen
format. The following ``template'' .smesh file is the shortest way to
import those meshes into a .smesh file:
\\\\
\noindent{}*VERTICES\\
*INCLUDE myMesh.node\\
*ELEMENTS\\
TET\\
*INCLUDE myMesh.ele

\bibliographystyle{eg-alpha}
\bibliography{ref}

\end{document}